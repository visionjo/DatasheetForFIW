\section*{Data Collection Process}

% \end{center}
\subsection*{How was the data associated with each instance acquired?}
% Was the data directly observable (\eg raw text, movie ratings), reported by subjects ((\eg survey responses), or indirectly inferred/derived from other data (\eg part-of-speech tags, model-based guesses for age or language)? If data was reported by subjects or indirectly inferred/derived from other data, was the data validated/verified? If so, please describe how.

\noindent The names for each person in the dataset were determined by an
operator by looking at the caption associated with the person’s
photograph.

\subsection*{What mechanisms or procedures were used to collect the data (\eg hardware apparatus or sensor, manual human curating, software program, software API)?}
% How were these mechanisms or procedures validated?
\noindent The raw images for this dataset were obtained from the Faces in
the Wild database collected by Tamara Berg at Berkeley3. The images in this database were gathered from news articles on the web using software to crawl news articles.


\subsection*{If the dataset is a sample from a larger set, what was the sampling strategy (\eg deterministic, probabilistic with specific sampling probabilities)?}

\noindent The original Faces in the Wild dataset is a sample of pictures of people appearing in the news on the web. Labeled Faces in the
Wild is thus also a sample of images of people found on the news
on line. While the intention of the dataset is to have a wide range
of demographic (e.g. age, race, ethnicity) and image (e.g. pose,
illumination, lighting) characteristics, there are many groups that
have few instances (e.g. only 1.57\% of the dataset consists of
individuals under 20 years old).

\subsection*{Who was involved in the data collection process (\eg students, crowd-workers, contractors) and how were they compensated (\eg
how much were crowdworkers paid)?}

\noindent Students of NEU.

\subsection*{Over what time-frame was the data collected?}
% \texttt{Does this timeframe match the creation timeframe of the data associated with the instances (\eg recent crawl of old news articles)? If not, please describe the timeframe in which the data associated with the instances was created




\subsection*{Were any ethical review processes conducted (\eg by an institutional review board)?}
% If so, please provide a description of these review processes, including the outcomes, as well as a link or other access point to any supporting documentation.
\noindent Unknown


\subsection*{Does the dataset relate to people?}
% If not, you may skip the remaining questions in this section.
\noindent Yes. Each instance is an image of a person.

\subsection*{Did you collect the data from the individuals in question directly, or obtain it via third parties or other sources (\eg websites)?}
\noindent The data was crawled from public web sources.

\subsection*{Were the individuals in question notified about the data collection?}
% \texttt{If so, please describe (or show with screenshots or other information) how notice was provided, and provide a link or other access point to, or otherwise reproduce, the exact language of the notification itself.}}
\noindent Unknown

\subsection*{Did the individuals in question consent to the collection and use of their data?}
%If so, please describe (or show with screenshots or other information) how consent was requested and provided, and provide a link or other access point to, or otherwise reproduce, the exact language to which the individuals consented.

\noindent No. All subjects in the dataset appeared in news sources so the
images that we used along with the captions are already public.

\subsection*{If consent was obtained, were the consenting individuals provided with a mechanism to revoke their consent in the future or for certain uses?}
% If so, please provide a description, as well as a link or other access point to the mechanism (if appropriate).
\noindent No. The data was crawled from public web sources, and the individuals appeared in news stories. But there was no explicit informing of these individuals that their images were being assembled into a dataset.

\subsection*{Has an analysis of the potential impact of the dataset and its use on data subjects (\eg a data protection impact analysis)been conducted?}
%  If so, please provide a description of this analysis, including the outcomes, as well as a link or other access point to any supporting documentation.
\noindent Unknown

\subsection*{Any other comments?}
\noindent 

\subsection*{\texttt{d}}
\noindent 

\subsection*{How was the data collected? (\eg hardware apparatus/sensor, manual human curating, software program, software interface/API)}
\noindent


\subsection*{Who was involved in the data collection process? (\eg students, crowd-workers) and how were they compensated (\eg how much were crowd-workers paid)?}
\noindent Several student volunteers, along with experts overseeing many of the details. No monetary cost was acquired for this effort.

\subsection*{Over what time-frame was the data collected? Does the collection time-frame match the creation time-frame of the instances?}
\noindent
\fls{fiw} was created over a span of months. The time-frame of data (\ie approximately when the images were captured) does not match that of its creation-- imagery was scraped from the web and varies in date, which is metadata we do not have (\ie no evidence or time stamp indicating its originality).


\subsection*{Does the dataset contain all possible instances? Or is it a sample (not necessarily random) of instances from a larger set?}
\noindent
\subsection*{If the dataset is a sample, then what is the population? What was the sampling strategy (\eg deterministic, probabilistic with specific sampling probabilities)? Is the sample representative of the larger set (\eg geographic coverage)? If not, why not (\eg to cover a more diverse range of instances)? How does this affect possible uses?}
\noindent
\subsection*{Is there information missing from the dataset and why? (this does not include intentionally dropped instances; it might include, \eg redacted text, withheld documents) Is this data missing because it was unavailable?}
\noindent
An abundance of faces have been added, but are not released publicly at this moment. The reason is directly related to the lack of public benchmark with lists including the newest face data. Nonetheless, the added data can be provided upon request; also, updated benchmarks will be released as a part of future work.

